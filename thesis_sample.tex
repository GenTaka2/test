\documentclass[11pt]{jreport}
\usepackage{wusepre_thesis}
\usepackage{indentfirst}
\usepackage{url}	% \url{}コマンド用.URLを表示する際に便利
%\usepackage{graphicx}  % ←graphicx.styを用いてEPSを取り込む場合有効にする
			% 他のパッケージ・スタイルを使う場合には適宜追加

%%%%%%%%%%%%%%%%%%%%%%%%%%%%%%%%%%%%%%%%%%%%%%%%%%%%%%%%%%%%%%%%%%%%%%%%

%%
%% 主に表紙を作成するための情報
%%

%%  タイトル(修論の場合は英語表記も指定)
\title{任意方向のカメラ移動を伴うモーションキャプチャ \\
        及び端末の自己位置推定の手法}
%\etitle{Test\\Test\\Test}

%%  著者名(修論の場合は英語表記も指定)
\author{高野 源太}
%\eauthor{Naoki Fukuyasu}

%% 卒業論文・修士論文(以下のどちらかを選択)
\prebachelar % プレ卒論(3年生用)
%\bachelar	% 卒業論文(4年生用)
%\master  	% 修士論文(M2用)

%%  学科・クラスタ
\department{システム工}
%\department{デザイン情報}
%\department{デザイン科学}

%%  学生番号
\studentid{60246336}

%%  卒業年度
\gyear{2021}		% 提出年が2021年なら,2020年度

%%  論文提出日
\date{2022年2月10日}	% 修士の場合は月(2021年2月)までとし,英語表記も指定
%\edate{February 2021}	% 修士の場合,こちら(英語表記)も有効化

%%%%%%%%%%%%%%%%%%%%%%%%%%%%%%%%%%%%%%%%%%%%%%%%%%%%%%%%%%%%%%%%%%%%%%%%

\begin{document}

\maketitle

%%
%%  概要
%%
\begin{abstract}


\end{abstract}

%%  目次
\tableofcontents

%%  図目次 (図目次をいれたければ以下のコメントをはずす)
%\listoffigures

%%  表目次 (表目次をいれたければ以下のコメントをはずす)
%\listoftables

\newpage
\pagenumbering{arabic}	% 以降のページ番号を算用数字に

%%%%%%%%%%%%%%%%%%%%%%%%%%%%%%%%%%%%%%%%%%%%%%%%%%%%%%%%%%%%%%%%%%%%%%%%

%%
%%  本文はここから
%%

\chapter{はじめに}
現在のスマホなどの単眼カメラが搭載された端末で行う光学式モーションキャプチャはカメラ位置を固定することが前提となっているものがほとんどであり、端末を動かしながらモーションキャプチャを行うといった利用方法は想定されていない。そのため、部屋内での端末の位置を推定し、端末の移動・カメラの向き等を考慮したモーションキャプチャの手法を確立することで、カメラ画角等にとらわれないモーションキャプチャを行うことができ、また、より簡易的にモーションキャプチャを利用することができるのではないかと考えた。

この研究では、部屋等の閉空間において一人の人間がカメラを持ちながらモーションキャプチャを行うといったような、単眼カメラの移動を考慮したモーションキャプチャの手法について検討する。通常モーションキャプチャは

\chapter{関連研究}


\chapter{使用機材・実装環境}


\chapter{提案手法}\label{chap:fig-tab-exp}


\chapter{参考文献}


%%%%%%%%%%%%%%%%%%%%%%%%%%%%%%%%%%%%%%%%%%%%%%%%%%%%%%%%%%%%%%%%%%%%%%%%

%%
%% 謝辞
%%
%% \begin{acknowledgements}
%% 感謝します.
%% \end{acknowledgements}

%%%%%%%%%%%%%%%%%%%%%%%%%%%%%%%%%%%%%%%%%%%%%%%%%%%%%%%%%%%%%%%%%%%%%%%%

%%
%% 参考文献
%%
\begin{thebibliography}{99}

\bibitem{wusethesis}
  福安直樹,
  卒業論文スタイルファイル(和歌山大学システム工学部用),\\
  \url{https://github.com/fukuyasu/wuse_thesis}.


\end{thebibliography}

%%%%%%%%%%%%%%%%%%%%%%%%%%%%%%%%%%%%%%%%%%%%%%%%%%%%%%%%%%%%%%%%%%%%%%%%

%%
%% 付録
%%
% \appendix
% 
% \chapter{サンプルプログラム}
% 
% プログラムリストや実行結果など,本論を補足する上で必要と思われるものが
% あれば付録として付ける.
% 
% {
% \footnotesize
% \begin{verbatim}
% #include <stdio.h>
% int main(void)
% {
%     printf("Hello, World!\n");
%     return 0;
% }
% \end{verbatim}
% }

%%%%%%%%%%%%%%%%%%%%%%%%%%%%%%%%%%%%%%%%%%%%%%%%%%%%%%%%%%%%%%%%%%%%%%%%

\end{document}
